\documentclass[12pt]{article}

\usepackage{graphicx}
\usepackage{paralist}
\usepackage{amsfonts}
\usepackage{amsmath}
\usepackage{hhline}
\usepackage{booktabs}
\usepackage{multirow}
\usepackage{multicol}
\usepackage{url}

\oddsidemargin -10mm
\evensidemargin -10mm
\textwidth 160mm
\textheight 200mm
\renewcommand\baselinestretch{1.0}

\pagestyle {plain}
\pagenumbering{arabic}

\newcounter{stepnum}

%% Comments

\usepackage{color}

\newif\ifcomments\commentstrue

\ifcomments
\newcommand{\authornote}[3]{\textcolor{#1}{[#3 ---#2]}}
\newcommand{\todo}[1]{\textcolor{red}{[TODO: #1]}}
\else
\newcommand{\authornote}[3]{}
\newcommand{\todo}[1]{}
\fi

\newcommand{\wss}[1]{\authornote{blue}{SS}{#1}}

\title{Assignment 4, Specification}
\author{SFWR ENG 2AA4}

\begin {document}

\maketitle
This Module Interface Specification (MIS) document contains modules, types and
methods for implementing the state of a cellular automaton of Conway's Game of Life.

\newpage

\section* {Game Types Module}

\subsection*{Module}

GameTypes

\subsection* {Uses}

N/A

\subsection* {Syntax}

\subsubsection* {Exported Constants}

BOARDLENGTH = 25

\subsubsection* {Exported Types}

CellState = $\{\text{Dead, Alive}\} \\ $

\subsubsection* {Exported Access Programs}

None

\subsection* {Semantics}

\subsubsection* {State Variables}

None

\subsubsection* {State Invariant}

None

\newpage

\section* {Game Board ADT Module}

\subsection*{Template Module}

GridT

\subsection* {Uses}

\noindent BoardTypes\\

\subsection* {Syntax}

\subsubsection* {Exported Access Programs}

\begin{tabular}{| l | l | l | l |}
\hline
\textbf{Routine name} & \textbf{In} & \textbf{Out} & \textbf{Exceptions}\\
\hline
new GridT  & seq of (seq of CellState) & GridT & t\\
\hline
checkCell & $\mathbb{N}$, $\mathbb{N}$ & CellState & \\
\hline
update & & GridT & \\
\hline
neighbourCnt & GridT, $\mathbb{N}$, $\mathbb{N}$ & $\mathbb{N}$ & invalid\_argument\\
\hline
stateCount & CellState & $\mathbb{N}$ & \\
\hline
init\_seq & $\mathbb{N}$ & seq of (seq of CellState) & invalid\_argument \\
\hline
\end{tabular}

\subsection* {Semantics}

\subsubsection* {State Variables}

$Grid$: seq of (seq of CellState) \textit{\# Grid}\\


\subsubsection* {State Invariant}

stateCount(Alive) + stateCount(Dead) = BOARDLENGTH x BOARDLENGTH

\subsubsection* {Assumptions \& Design Decisions}

\begin{itemize}

\item The GridT constructor is called before any other access
  routine is called on that instance. Once a BoardT has been created, the
  constructor will not be called on it again.

\item The edge of the board will process non-existent cells on its border as dead cells.

\item For better scalability, this module is specified as an Abstract Data Type
  (ADT) instead of an Abstract Object. This would allow multiple games to be
  created and tracked at once by a client.

\item The getter function is provided to obtain the state of a cell and will violate the property of being essential. Regardless, this function is needed to obtain information about the game and ensure the ease of implementation. It is used in the view module as well as part of the Model View Controller design pattern implementation.
  (\url{https://blog.codinghorror.com/understanding-model-view-controller/})

\item The length of the grid at 25 which is used to create a set 25x25 grid, similar to how the dimensions of a board game are determined by the designer. The input/output files will follow this constraint. 

\end{itemize}

\subsubsection* {Access Routine Semantics}

\noindent \textsl{Constructor for the grid}
\noindent GridT(grid):
\begin{itemize}
\item transition: 
$Grid := grid$
\item output:
$out := self$
\end{itemize} 

\noindent \textsl{Updates the grid for the next stage iteration}\\
\noindent update():
\begin{itemize}
\item output: 
$out := (\forall x, y : \mathbb{N} | x, y \in [0...24] : updateCell(x, y)) $
\end{itemize}

\noindent \textsl{Returns the state of the cell} \\
\noindent checkCell($n_1, n_2$):
\begin{itemize}
\item output:
$out := Grid[n_1][n_2]$
\end{itemize}

\noindent \textsl{Returns the number of neighbours} \\
\noindent neighbourCnt(grid, $n_1, n_2$):
\begin{itemize}
\item output:
$out := +(c : CellState, x : \mathbb{N}, y : \mathbb{N} | x, y \in [-1..2] \land (x \neq 0 \land y \neq 0) \land c = Alive \implies Grid[n_1 + x][n_2 + y] = c : 1) $
\item exception: ($n_1 < 0 \lor n_1 > \text{BOARDLENGTH} \lor n_2 < 0 \lor n_2 > \text{BOARDLENGTH} \implies$ invalid\_argument)
\end{itemize}

\noindent \textsl{Returns the number of alive/dead states}\\
\noindent stateCount(state):
\begin{itemize}
\item output:
$out := (+n_1 : \mathbb{N}, n_2 : \mathbb{N} | n_1, n_2 \in [0...24] \land checkCell(n_1, n_2) = state: 1)$
\end{itemize}

\noindent \textsl{Creates an empty grid}\\
\noindent init\_seq($n$):
\begin{itemize}
\item output: $out := s \text{ such that } (|s| = n \land (\forall\ c : CellState | i, j
\in [0..n-1] \land c = Dead: s[i][j] = c)$
\item exception: n $\neq$ BOARDLENGTH $\implies$ invalid\_argument \\
\end{itemize}

\subsection*{Local Types}

N/A

\subsection*{Local Functions}

\noindent \textsl{Updates a cell according to its neighbours}\\
\noindent updateCell(x, y): $\mathbb{N} \times \mathbb{N} \rightarrow \mathbb{B}$\\
\noindent updateCell(x, y) $\equiv$
\begin{itemize}
\item transition: \\
\begin{tabular}{|p{4cm}|p{3.5cm}|l|}
\hhline{|-|-|-|}
cellState(x, y) = Alive & neighbourCnt(x, y) $<$ 2 $\lor$ neighbourCnt(x, y) $> $ 3  & Grid[x][y] = Dead \\
\hhline{|~|-|-|}
& neighbourCnt(x, y) = 2 $\lor$ neighbourCnt(x, y) = 3 & Grid[x][y] = Alive \\
\hhline{|-|-|-|}
cellState(x, y) = Dead & neighbourCnt(x, y) = 3 & Grid[x][y] = Alive\\
\hhline{|~|-|-|}
& neighbourCnt(x, y) $\neq$ 3  & Grid[x][y] = Dead \\
\hhline{|-|-|-|}
\end{tabular}\\\\

\item output: $out := True$
\end{itemize}
\newpage

\section* {View Module}

\subsection*{Template Module}

GridT

\subsection* {Uses}

\noindent BoardTypes, GameBoard\\

\subsection* {Syntax}

\subsubsection* {Exported Access Programs}


\begin{tabular}{| l | l | l | l |}
\hline
\textbf{Routine name} & \textbf{In} & \textbf{Out} & \textbf{Exceptions}\\
\hline
Read  & String & GridT & \\
\hline
show & GridT &  & \\
\hline
write & GridT & &\\
\hline
\end{tabular}

\subsection* {Semantics}

\subsubsection* {Environment Variables}

readboard: A file containing a gameboard's initial state
writeboard: A file showing a current gameboard state.

\subsubsection* {State Variables}

\subsubsection* {State Invariant}

stateCount(Alive) + stateCount(Dead) = BOARDLENGTH x BOARDLENGTH

\subsubsection* {Assumptions \& Design Decisions}
\begin{itemize}
\item The Read()'s inout file will be formatted correctly.
\end{itemize}
\subsubsection* {Access Routine Semantics}

\noindent Read(file):
\begin{itemize}
\item output:
Read data from the file and initialize a GridT that will act as a starting gameboard for the module. The file contains 25x25 square brackets with either a blank space or a * in between. The blank represents a dead cell and the * represents a live cell. Takes this data and create a matching GridT ADT. 
\item exception: None
\end{itemize}


\noindent Show(grid):
\begin{itemize}
\item output:
Displays the input grid on the console with the same format as the input file. A 25x25 board containing square brackets with a blank or * to represent dead or alive cells, respectively.
\item exception: None 
\end{itemize}

\noindent Write(grid):
\begin{itemize}
\item output:
Writes the input grid out to an output file with the same format as the input file. A 25x25 board containing square brackets with a blank or * to represent dead or alive cells, respectively. 
\item exception: None
\end{itemize}

\newpage

\section* {Critique of Design}

The design of this game was split into two main portions: the GameBoard and the View Module. As well, a BoardTypes module was constructed to outline two major facts about the board: the length and the cell types. The reason these modules were split were to outline the Model and the View portions of the Model-View-Controller implementation required. For the design, some decisions and assumptions were made to implement the design. First, the border cells treated the imaginary adjacent cells off the board as dead cells. This was made do that only live cells on the board would affect the game. Also, the file format was chosen to go hand-in-hand with the ASCII character display for continuity. More so, the instances created all specifically allocated tasks to increase readability and code organization. 

The design is consistent across the different modules and the MIS. The MIS's outlines are followed by the code effectively and the requirements are outlined to correctly represent the program. As stated before, the file formats and display were consistent with one another. It is crucial for the formats of the file to be similar as it is not only required, but the code processes them similarly.

The design violated some aspects of essentiality as code organization and ease of use were prioritized. For example, I could have gathered and created an empty sequence without the extra function call of init\_seq. Instead, this instance was created for better organization and increased cohesion. Minimality was not violated since every function served only one purpose and could not be split into two tasks. The design also respected information hiding since the state variables could not be accessed externally and instances were privatized, like init\_seq. Generality was ignored in a sense through the GameBoard module when only a sequence of a sequence of CellStates was created. The module would have been general to accommodate all types representing Alive or Dead, but the module was limited to Cellstate for consistency. Only this type could be used in the implementation of the game.

Lastly, the code was cohesive and well-structured as the set of instances in both modules are closely related to one another. For example, in the GameBoard module, all instances are accessing the cells and either obtaining similar information or modifying its state. In the View Module, the tree functions are processing the exact same file/display format, with read and write performing identical but opposing functions. 

 


\end {document}